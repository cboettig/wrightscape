\documentclass[author-year, review, 12pt]{elsarticle} %review=doublespace preprint=single 5p=2 column
\usepackage{amsmath, amsfonts, amssymb}  % extended mathematics
% My package additions
\usepackage[hyphens]{url}
\usepackage{lineno} % add 
\linenumbers % turns line numbering on 
\bibliographystyle{elsarticle-harv}
\biboptions{sort&compress} % For natbib
\usepackage{graphicx}
\usepackage{booktabs} % book-quality tables

%% Redefines the elsarticle footer
\makeatletter
\def\ps@pprintTitle{%
 \let\@oddhead\@empty
 \let\@evenhead\@empty
 \def\@oddfoot{\it \hfill\today}%
 \let\@evenfoot\@oddfoot}
\makeatother

% A modified page layout
\textwidth 6.75in
\oddsidemargin -0.15in
\evensidemargin -0.15in
\textheight 9in
\topmargin -0.5in


\usepackage{microtype}
\usepackage{fancyhdr}
\pagestyle{fancy}
\pagenumbering{arabic}

\usepackage{listings}
\lstnewenvironment{code}{\lstset{language=Haskell,basicstyle=\small\ttfamily}}{}


\setlength{\parindent}{0pt}
\setlength{\parskip}{6pt plus 2pt minus 1pt}


%%% Syntax Highlighting for code  %%%
%%% Adapted from knitr book %%% 
\usepackage{fancyvrb}
\DefineVerbatimEnvironment{Highlighting}{Verbatim}{commandchars=\\\{\}}
% Add ',fontsize=\small' for more characters per line
\newenvironment{Shaded}{}{}
\newcommand{\KeywordTok}[1]{\textcolor[rgb]{0.00,0.44,0.13}{\textbf{{#1}}}}
\newcommand{\DataTypeTok}[1]{\textcolor[rgb]{0.56,0.13,0.00}{{#1}}}
\newcommand{\DecValTok}[1]{\textcolor[rgb]{0.25,0.63,0.44}{{#1}}}
\newcommand{\BaseNTok}[1]{\textcolor[rgb]{0.25,0.63,0.44}{{#1}}}
\newcommand{\FloatTok}[1]{\textcolor[rgb]{0.25,0.63,0.44}{{#1}}}
\newcommand{\CharTok}[1]{\textcolor[rgb]{0.25,0.44,0.63}{{#1}}}
\newcommand{\StringTok}[1]{\textcolor[rgb]{0.25,0.44,0.63}{{#1}}}
\newcommand{\CommentTok}[1]{\textcolor[rgb]{0.38,0.63,0.69}{\textit{{#1}}}}
\newcommand{\OtherTok}[1]{\textcolor[rgb]{0.00,0.44,0.13}{{#1}}}
\newcommand{\AlertTok}[1]{\textcolor[rgb]{1.00,0.00,0.00}{\textbf{{#1}}}}
\newcommand{\FunctionTok}[1]{\textcolor[rgb]{0.02,0.16,0.49}{{#1}}}
\newcommand{\RegionMarkerTok}[1]{{#1}}
\newcommand{\ErrorTok}[1]{\textcolor[rgb]{1.00,0.00,0.00}{\textbf{{#1}}}}
\newcommand{\NormalTok}[1]{{#1}}
\usepackage{enumerate}
\usepackage{ctable}
\usepackage{float}

% This is needed because raggedright in table elements redefines \\:
\newcommand{\PreserveBackslash}[1]{\let\temp=\\#1\let\\=\temp}
\let\PBS=\PreserveBackslash
\usepackage[normalem]{ulem}
\newcommand{\textsubscr}[1]{\ensuremath{_{\scriptsize\textrm{#1}}}}

% Configure hyperlinks package
\usepackage[breaklinks=true,linktocpage,pdftitle={Detecting a release of constraint in Labrid fish},pdfauthor={Carl Boettiger, Jeremy Beaulieu and Peter C. Wainwright},xetex,colorlinks]{hyperref}
\hypersetup{breaklinks=true, pdfborder={0 0 0}}

% Pandoc toggle for numbering sections (defaults to be off)
\setcounter{secnumdepth}{0}


\VerbatimFootnotes % allows verbatim text in footnotes

% Pandoc header



\begin{document}
\begin{frontmatter}
  \title{Detecting a release of constraint in Labrid fish}
  \author[cpb]{Carl Boettiger\corref{cor1}}
  \author[cpb]{Peter C. Wainwright}
  \ead{cboettig@ucdavis.edu}
  \cortext[cor1]{Corresponding author, cboettig@ucdavis.edu}
  \address[cpb]{Center for Population Biology, University of California, Davis, California 95616}
 \end{frontmatter}


\section{Introduction}

\subsection{The release of constraint hypothesis}

\subsection{Key innovations in parrotfish}

(Price et al. 2010)

\begin{figure}[htbp]
\centering
\includegraphics{figure/labrid_phylo.pdf}
\caption{The phylogenetic tree of Labrid fish used in this study. We
divide the tree into three clades: Wrasses, Parrotfish with an
intramandibular joint and pharyngeal joint, and parrotfish that lack the
intramandibular joint. Diagrams of the jaw structure for representative
species from each group are shown adjacent.}
\end{figure}

\subsection{Models for a release of constraint}

The Ornstein-Uhlenbeck (OU) process is a stochastic, mean-reverting
process commonly used in the comparative phylogenetics context to model
the evolution of a trait under constraint (Hansen and Martins 1996). The
model has been generalized to the case of multiple optima in (Butler and
King 2004) and recently to the case of differing strengths of selection
(Beaulieu et al. 2012). This latest extension is perhaps most
interesting when used to detect a change in selection strength in a
sub-clade following a potential innovation. This would predict and
increase in the rate disparity increases in some focal traits in the
sub-clade relative to the base rate observed. We will refer to this
model in which the strength of stabilizing selection decreases as the
release of constraint model.

Unfortunately, this basic pattern corresponds to a similar scenario that
does not involve a release of constraint. It is commonly postulated that
a trait's evolutionary pattern may correspond best to a Brownian motion
(BM) process without no central tendency imposed by the constraint in
the OU model. If the basic Brownian rate parameter increases at the time
of the innovation, a pattern of increased growth in disparity can still
be observed without the corresponding mechanism of a release of
constraint. Modeling a change in a Brownian rate parameter was first
introduced in O'Meara et al. (2006) in the software \emph{Brownie};
hence we will refer to this as the Brownie model.

Though the processes are not identical, the exhibit remarkably similar
patterns. Figure 2 illustrates the Brownie and release of constraint
models through 500 replicate simulations of each. In the top panel, BM
and OU processes with comparable parameters are shown for reference. In
both the Brownie and release of constraint models, the same variance has
been reached at the time of the shift and at the time the simulation
ends. The distinguishing feature in the release of constraint model is
similar to the distinction between a BM and OU processes -- before the
shift occurs, the trait dynamics have begun to approach an equilibrium
that balances the diversification process against the constraint. This
corresponds to traits values more closely reflecting a match to their
environment than to their evolutionary history. After the shift in
selection occurs, the traits begin to explore outside the range
previously possible under the strong constraint.

\begin{figure}[htbp]
\centering
\includegraphics{figure/figure2.pdf}
\caption{}
\end{figure}

By contrast, the Brownie model shows a sharper transition at this
boundary. A close look at the figure shows a shock front at the moment
of this transition.\\Whereas in the release of constraint model, the
trajectories only loose their central bias, but otherwise continue to
make similar step sizes, in the Brownie model the entire tempo of the
evolutionary process has changed, taking bigger steps in both
directions. It is these subtle differences in the patterns of the
evolutionary processes implied by the different models that we seek to
tease apart. To obtain the most powerful statistical comparison between
the two models that accounts for the uncertainty in the model estimate,
we use the method described in Boettiger, Coop, and Ralph (2012) which
uses a bootstrap simulation approach to compare likelihood ratios of the
models.

\section{Results}

\subsection{Estimating the release of constraint model}

\begin{figure}[htbp]
\centering
\includegraphics{figure/figure3.pdf}
\caption{}
\end{figure}

\subsection{Comparing models}

\begin{figure}[htbp]
\centering
\includegraphics{figure/figure4.pdf}
\caption{}
\end{figure}

(Show that Brownie model rejects the OUCH model/thetas?)

\section{Discussion}

Our analysis identified two functional traits in the labrid jaw
morphology that show clear evidence of a release of constraint.

Functional innovations\ldots{}

\section{Acknowledgements}

This work was supported by a Computational Sciences Graduate Fellowship
from the Department of Energy under grant number DE-FG02-97ER25308 to CB
and NSF grant DEB-1061981 to PCW.

\section{References}

Beaulieu, Jeremy M., Dwueng-Chwuan Jhwueng, Carl Boettiger, and Brian C.
O'Meara. 2012. ``Modeling Stabilizing Selection: Expanding the
Ornstein-Uhlenbeck Model of Adaptive Evolution.'' \emph{Evolution}
(mar). doi:10.1111/j.1558-5646.2012.01619.x.
\href{http://doi.wiley.com/10.1111/j.1558-5646.2012.01619.x}{http://doi.wiley.com/10.1111/j.1558-5646.2012.01619.x}.

Boettiger, Carl, Graham Coop, and Peter Ralph. 2012. ``Is your phylogeny
informative? Measuring the power of comparative methods.''
\emph{Evolution} (jan). doi:10.1111/j.1558-5646.2012.01574.x.
\href{http://doi.wiley.com/10.1111/j.1558-5646.2012.01574.x}{http://doi.wiley.com/10.1111/j.1558-5646.2012.01574.x}.

Butler, Marguerite A., and Aaron A. King. 2004. ``Phylogenetic
Comparative Analysis: A Modeling Approach for Adaptive Evolution.''
\emph{The American Naturalist} 164 (dec): 683--695. doi:10.1086/426002.
\href{http://www.jstor.org/stable/10.1086/426002}{http://www.jstor.org/stable/10.1086/426002}.

Hansen, Thomas F., and E. P. Martins. 1996. ``Translating between
microevolutionary process and macroevolutionary patterns: the
correlation structure of interspecific data.'' \emph{Evolution} 50:
1404--1417.

O'Meara, Brian C., Cécile Ané, Michael J. Sanderson, and Peter C.
Wainwright. 2006. ``Testing for different rates of continuous trait
evolution using likelihood.'' \emph{Evolution} 60 (may): 922--33.
\href{http://www.ncbi.nlm.nih.gov/pubmed/16817533}{http://www.ncbi.nlm.nih.gov/pubmed/16817533}.

Price, Samantha a, Peter C. Wainwright, David R. Bellwood, Erem
Kazancioglu, David C. Collar, and Thomas J. Near. 2010. ``Functional
innovations and morphological diversification in parrotfish.''
\emph{Evolution; international journal of organic evolution} 64 (oct):
3057--68. doi:10.1111/j.1558-5646.2010.01036.x.
\href{http://www.ncbi.nlm.nih.gov/pubmed/20497217}{http://www.ncbi.nlm.nih.gov/pubmed/20497217}.


\bibliography{}


\end{document}
