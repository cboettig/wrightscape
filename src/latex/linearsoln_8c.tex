\hypertarget{linearsoln_8c}{
\section{linearsoln.c File Reference}
\label{linearsoln_8c}\index{linearsoln.c@{linearsoln.c}}
}
{\tt \#include \char`\"{}tree.h\char`\"{}}\par
\subsection*{Functions}
\begin{CompactItemize}
\item 
double \hyperlink{linearsoln_8c_72563a25bf023e3c976c2a8ab861a5c3}{sep\_\-time} (int i, int j, tree $\ast$mytree)
\item 
double \hyperlink{linearsoln_8c_387c8e2634f15e3d68e4b4371f28e78a}{ou\_\-likelihood} (tree $\ast$mytree)
\item 
\hypertarget{linearsoln_8c_ad3bc1a98d03589221bd0be337d8a074}{
double \textbf{bm\_\-likelihood} (tree $\ast$mytree)}
\label{linearsoln_8c_ad3bc1a98d03589221bd0be337d8a074}

\item 
double \hyperlink{linearsoln_8c_3041e81cc6c05acd43545a9a59ccb16d}{bm\_\-ancestral\_\-likelihood} (tree $\ast$mytree)
\end{CompactItemize}


\subsection{Detailed Description}
\begin{Desc}
\item[Author:]Carl Boettiger, $<$\href{mailto:cboettig@gmail.com}{\tt cboettig@gmail.com}$>$ \end{Desc}
\hypertarget{linearsoln_8c_DESCRIPTION}{}\subsection{DESCRIPTION}\label{linearsoln_8c_DESCRIPTION}
Note: Comments formatted for Doxygen, which generates the documentation files.

The likelihood function under a linear model is a multivariate normal. In the case of Brownian Motion, it is

\[ L(\mu, \sigma^2) \propto \frac{1}{\sigma^{2N}} \exp\left( \frac{-Q(\vec \mu)}{2\sigma^2}\right) \] \[ Q(\vec \mu ) = \sum_{i,j} \frac{(\mu_i - mu_j)^2}{\nu_{ij} } \]

Where $\vec \mu = \{ \mu_1 \ldots \mu_N\} $ are the traits at the nodes and $\nu_{ij} $ the distances between them. 

\subsection{Function Documentation}
\hypertarget{linearsoln_8c_3041e81cc6c05acd43545a9a59ccb16d}{
\index{linearsoln.c@{linearsoln.c}!bm\_\-ancestral\_\-likelihood@{bm\_\-ancestral\_\-likelihood}}
\index{bm\_\-ancestral\_\-likelihood@{bm\_\-ancestral\_\-likelihood}!linearsoln.c@{linearsoln.c}}
\subsubsection[{bm\_\-ancestral\_\-likelihood}]{\setlength{\rightskip}{0pt plus 5cm}double bm\_\-ancestral\_\-likelihood (tree $\ast$ {\em mytree})}}
\label{linearsoln_8c_3041e81cc6c05acd43545a9a59ccb16d}


Likelihood under BM of an ancestral state configuration (all internal as well as tip nodes) \hypertarget{linearsoln_8c_387c8e2634f15e3d68e4b4371f28e78a}{
\index{linearsoln.c@{linearsoln.c}!ou\_\-likelihood@{ou\_\-likelihood}}
\index{ou\_\-likelihood@{ou\_\-likelihood}!linearsoln.c@{linearsoln.c}}
\subsubsection[{ou\_\-likelihood}]{\setlength{\rightskip}{0pt plus 5cm}double ou\_\-likelihood (tree $\ast$ {\em mytree})}}
\label{linearsoln_8c_387c8e2634f15e3d68e4b4371f28e78a}


The likelihood of a model under the OU process, \[ dX_t = \alpha (\theta - X_t) \d t + \sigma \d W_t \] is multivariate normal. If only the n tips are specified (not computing likelihood of ancestral state) then this is determined by the n mean values and the n x n covariance matrix. The expected value at any node is determined only by the age of the node and the intitial condition of the root node ($ X_0 $): \[ E(X_t | X_0 ) = X_0 e^{-\alpha t} + \theta (1- e^{-\alpha t}) \] The i,j element of the covariance matrix is \[ V_{ij} = \frac{\sigma^2}{2\alpha} (1 - e^{-2 \alpha s_{ij} }) e^{-2\alpha (t-s_{ij} )} \] where the nodes are all measured at time t (present day) and have diverged for a time $ t-s_{ij} $ (thus shared time $ s_{ij} $ since $ X_0 $.

As this is multivariate normal, the the log-likelihood is then \[ \log L = (X - E(X) )^T V^{-1} (X-E(X) ) + N\log(2\pi \det V) \] 

$ E(X_t | X_0 ) = X_0 e^{-\alpha t} + \theta (1- e^{-\alpha t}) $ \hypertarget{linearsoln_8c_72563a25bf023e3c976c2a8ab861a5c3}{
\index{linearsoln.c@{linearsoln.c}!sep\_\-time@{sep\_\-time}}
\index{sep\_\-time@{sep\_\-time}!linearsoln.c@{linearsoln.c}}
\subsubsection[{sep\_\-time}]{\setlength{\rightskip}{0pt plus 5cm}double sep\_\-time (int {\em i}, \/  int {\em j}, \/  tree $\ast$ {\em mytree})}}
\label{linearsoln_8c_72563a25bf023e3c976c2a8ab861a5c3}


The branch length between nodes i and j; (twice their divergence time; down to common ancestor and back). Currently this requires i \& j to be the same age (i.e. both are tips in an ultrametric tree). The code opens with a warning message to handle this.

This should really implement a clever and general algorithm. See the wikipedia page on Least Common Ancestor or the tutorial \begin{Desc}
\item[See also:]\{\href{http://www.topcoder.com/tc?module=Static&d1=tutorials&d2=lowestCommonAncestor}{\tt http://www.topcoder.com/tc?module=Static\&d1=tutorials\&d2=lowestCommonAncestor}\} \end{Desc}
